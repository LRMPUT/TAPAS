\documentclass[a4paper,12pt]{article}

\usepackage[body={160mm,240mm},top=25mm,left=25mm,marginparwidth=0mm,marginparse
p=0mm,nohead]{geometry}
\usepackage[T1]{fontenc}
\usepackage{hyperref}
\usepackage{graphicx}
\usepackage[cp1250]{inputenc}
%\usepackage{english}
%\usepackage[english]{babel}
\usepackage{times}
\usepackage{amsmath}
\usepackage{algorithm}
\usepackage{algpseudocode}
\usepackage{graphicx}
%\usepackage[norefs, nocites]{refcheck}
\usepackage{url}
\overfullrule=0pt
%\prefixing

%\floatname{algorithm}{Algorithm}

\pagestyle{empty}

\begin{document}

\title{TAPAS Project Report}
\author{Adam Bondyra, Micha� Nowicki, Jan Wietrzykowski}
\date{\today}
\maketitle

\section{Introduction}

\section{Mechanics}

\section{Electronics}

\section{Software}

\subsection{PositionEstimation}

\subsection{MovementConstraints}

\subsection{LocalPlanner}

\subsection{GlobalPlanner}

\section{Conclusion and future work}

\begin{figure}[h]
  \centering
  \includegraphics[height = 4cm]{figures//sample.jpg}
  \label{fig:example}
  \caption{Example image}
\end{figure}


\begin{equation}
  p(\mathbf{y} | \mathbf{x}) = \frac{1}{Z(\mathbf{x})} \prod_{a = 1}^{A}{\Psi_a (\mathbf{y}_a, \mathbf{x}_a)}.
\end{equation}

$$ \mathbf{\theta}_E = \begin{bmatrix}
			  -0,5 \\
			  -0,5 \\
			  -0,5 \\
			\end{bmatrix} $$

\begin{algorithm}
  \caption{Pr�bkowanie Gibbsa}
  \label{alg:gibbs}
  \begin{algorithmic}[1]
    \State Przypisz $\mathbf{y}^{j + 1} \leftarrow \mathbf{y}^{j}$.
    \For{dla ka�dego $i \in \mathcal{V}$}
      \State Pr�bkuj $\mathbf{y}_i^{j + 1}$ z dystrybucji $p(y_i | \mathbf{y}_{\setminus i}, \mathbf{x})$.
    \EndFor
    \State Zwr�� $\mathbf{y}^{j + 1}$.
  \end{algorithmic}
\end{algorithm}

\begin{table}[h]
	\centering
	\caption{Wyniki przed zastosowaniem CRF}
	\label{tab:wyniki_przed}
	\begin{tabular}{|c|c|c|c|c|}
		\hline
		  & \multicolumn{4}{c|}{przewidywane warto�ci} \\ \hline
		rodzaj terenu & trawa & chodnik 1 & chodnik 2 & asfalt \\ \hline
		trawa	& 98,9\% & 1,1\% & 0\% & 0\% \\
		chodnik 1 & 14\% & 85,2\% & 0,8\% & 0\% \\ \hline
	\end{tabular}
\end{table}

\small
\begin{thebibliography}{99}
  
\bibitem{gibbs}
  B. Walsh, Markov Chain Monte Carlo and Gibbs Sampling, {\em Lecture Notes for EEB 581}, 2004,

\end{thebibliography}

\end{document}
